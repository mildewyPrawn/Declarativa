\documentclass[spanish,12pt,letterpaper]{article}

\usepackage[english]{babel}
\usepackage[table]{xcolor}
\usepackage[utf8]{inputenc}
\usepackage{amsmath}
\usepackage{amssymb}
\usepackage{authblk}
\usepackage{csquotes}
\usepackage{enumerate}
\usepackage{float}
\usepackage{forest}
\usepackage{geometry}
\usepackage{graphicx}
\usepackage{listings}
\usepackage{xcolor}

\geometry{
  a4paper,
  total={170mm,257mm},
  left=20mm,
  top=20mm,
}

\definecolor{codegreen}{rgb}{0,0.6,0}
\definecolor{codegray}{rgb}{0.5,0.5,0.5}
\definecolor{codepurple}{rgb}{0.58,0,0.82}
\definecolor{backcolour}{rgb}{0.95,0.95,0.92}

\lstdefinestyle{mystyle}{
  backgroundcolor=\color{backcolour},
  commentstyle=\color{codegreen},
  keywordstyle=\color{magenta},
  numberstyle=\tiny\color{codegray},
  stringstyle=\color{codepurple},
  basicstyle=\ttfamily\footnotesize,
  breakatwhitespace=false,
  breaklines=true,
  captionpos=b,
  keepspaces=true,
  numbers=left,
  numbersep=5pt,
  showspaces=false,
  showstringspaces=false,
  showtabs=false,
  tabsize=2
}

\lstset{style=mystyle}

\title{Programación declarativa. Tarea 2\\
  \Huge{The Imperative is Dark and Full of Terrors}}
\author{Juan Alfonso Garduño Solís\\
  Emiliano Galeana Araujo}
\affil{Facultad de ciencias, UNAM}
\date{Fecha de entrega: Lunes 24 de febrero de 2020}

\begin{document}

\maketitle

\section{Demostraciones de propiedades}
\begin{itemize}
\item
  \begin{lstlisting}[language=Haskell]
    sum . map double = double . sum\end{lstlisting}
    \begin{itemize}
    \item Caso base:
      \begin{lstlisting}[language=Haskell]
        sum . map double [] = double . sum []
        sum . [] = double . sum []
        0 = double . sum []
        sum [] = double sum []
        double . sum [] = double . sum []\end{lstlisting}
    \item Hipótesis
      \begin{lstlisting}[language=Haskell]
        sum . map double xs = double . sum xs\end{lstlisting}
      \item Paso inductivo
        \begin{itemize}
        \item [--] Por demostrar
          \begin{lstlisting}[language=Haskell]
            sum . map double (x:xs) = double . sum (x:xs)\end{lstlisting}
        \end{itemize}
        \begin{lstlisting}[language=Haskell]
          sum . map double (x:xs) = double . sum (x:xs)
          -- Definicion de aplicar double a la cabeza de x:xs.
          sum . [double x] ++ map double xs = double . sum (x:xs)
          -- Definicion de aplicar sum ([double x] es igual a [2*x]).
          2*x + sum . map double xs = double . sum (x:xs)
          -- Hipotesis.
          2*x + double . sum xs = double . sum (x:xs)
          -- Defincion de double inversa.
          double x + sum xs = double . sum (x:xs)
          -- Metemos x a la funcion sum.
          double . sum (x:xs) = double . sum (x:xs) \end{lstlisting}
    \end{itemize}

  \item
    \begin{lstlisting}[language=Haskell]
      sum . map sum = sum . concat \end{lstlisting}
    \begin{itemize}
    \item Caso base:
      \begin{lstlisting}[language=Haskell]
        sum . map sum [] = sum . concat []
        sum [] = sum . concat []
        0 = sum . concat []
        sum [] = sum . concat []
        sum . concat [] = sum . concat [] \end{lstlisting}
    \item Hipótesis
      \begin{lstlisting}[language=Haskell]
        sum . map sum xs = sum . concat xs \end{lstlisting}
    \item Paso inductivo
      \begin{itemize}
      \item [--] Por demostrar
        \begin{lstlisting}[language=Haskell]
          sum . map sum (x:xss) = sum . concat (x:xss) \end{lstlisting}
      \end{itemize}
      \begin{lstlisting}[language=Haskell]
        sum . map sum (x:xss) = sum . concat (x:xss)
        -- Defincion de aplicar map sum a la cabeza de x:xss. Definimos
        -- sum' x como el resultado de sum x.
        sum [sum' x] . map sum xss = sum . concat (x:xss)
        -- Definicion de aplicar sum a una lista con un
        -- elemento (sum' x).
        (sum' x) + sum . map xss = sum . concat (x:xss)
        -- Hipotesis.
        (sum' x) + sum . concat (xss) = sum . concat (x:xss)
        -- Metemos la suma a la funcion sum.
        sum [sum' x] . concat (xss) = sum . concat (x:xss)
        -- Como [sum' x] es el resultado de la aplicar sum a la
        -- lista x. Siendo x una lista, podemos hacer lo siguiente.
        sum . concat (x:xss) = sum . concat (x:xss)\end{lstlisting}
    \end{itemize}

  \item
    \begin{lstlisting}[language=Haskell]
      sum . sort = sum \end{lstlisting}
    \begin{itemize}
    \item Definicion sort:
      \begin{lstlisting}[language=Haskell]
        sort [] = []
        sort [x] = [x]
        sort xs = mezcla (sort f) (sort s)
        where (f,s) = parte xs\end{lstlisting}

        Suponemos que parte está bien definido y parte de (x:xs) regresa una
        tupla $(f,s)$ dónde $f ++ s = (x:xs) $

      \item Caso base:
        \begin{lstlisting}[language=Haskell]
          sum . sort [] = sum []
          sum [] = sum [] \end{lstlisting}
      \item Hipótesis
        \begin{lstlisting}[language=Haskell]
          sum . sort xs = sum xs \end{lstlisting}
      \item Paso inductivo
        \begin{itemize}
        \item [--] Por demostrar
          \begin{lstlisting}[language=Haskell]
            sum . sort (x:xs) = sum (x:xs)
            sum . mezcla (sort f) (sort s)= sum (x:xs)
            -- Por la propiedad demostrada en clase sabemos que
            -- sum . mezcla (xs ys) = sum (xs ys), entonces
            sum . (sort f) ++ (sort s) = sum (x:xs)
            -- Por la otra propiedad demostrada en clase sabemos
            --  que: sum (xs ++ ys) = sum xs + sum ys, entonces:
            sum (sort f) + sum (sort s) = sum (x:xs)
            -- aplicamos la hipotesis de induccion:
            sum f + sum s = sum (x:xs)
            -- Aplicamos la segunda propiedad demostrada en
            -- clase hacia el otro lado
            sum (f ++ s) = sum (x:xs)
            -- Sabemos que f ++ s = (x:xs) porque como se
            -- establecio al principio, parte esta bien definida y
            -- terminamos
            sum (x:xs) = sum (x:xs) \end{lstlisting}
        \end{itemize}
    \end{itemize}

\end{itemize}
Donde, \texttt{double} se define de la siguiente manera:
\begin{lstlisting}[language=Haskell]
  double :: Integer -> Integer
  double x = 2 * x\end{lstlisting}
  Y, \texttt{sum}, \texttt{map}, \texttt{sort} y \texttt{concat} son las definidas en
  el \texttt{Prelude}, de Haskell.

  \section{Función take}
  En Haskell la función \texttt{take n} toma los primeros n elementos de una lista,
  mientras que \texttt{drop n} regresa la lista sin los primeros n elementos de
  esta.\\
  \noindent Definiciones:
  \begin{itemize}
  \item[--] take:
    \begin{lstlisting}[language=Haskell]
      take _ [] = []
      take 0 (x:xs) = []
      take n (x:xs) = x:take n-1 xs\end{lstlisting}
    \item[--] drop:
      \begin{lstlisting}[language=Haskell]
        drop _ [] = []
        drop 0 xs = xs
        drop n (x:xs) = drop n-1 xs\end{lstlisting}
      \item[--] ++:
        \begin{lstlisting}[language=Haskell]
          ++ [] _ = _
          ++ (x:xs) l = x: xs ++ l\end{lstlisting}

        \item[--] map:
          \begin{lstlisting}[language=Haskell]
            map _ [] = []
            map f (x:xs) = f x : map f xs
          \end{lstlisting}


        \item[--] filter:
          \begin{lstlisting}[language=Haskell]
            filter _ [] = []
            filter p (x:xs)
            | p x       = x : filter p xs
            | otherwise = filter p xs\end{lstlisting}


          \item[--] concat:
            \begin{lstlisting}[language=Haskell]
              concat [] = []
              concat (xs:xss) = xs ++ concat xss\end{lstlisting}

  \end{itemize}
  \noindent Demuesrta o da un contraejemplo:
  \begin{itemize}
  \item
    \begin{lstlisting}[language=Haskell]
      take n xs ++ drop n xs = xs \end{lstlisting}
    \begin{itemize}
    \item Caso base: \\
      \begin{lstlisting}[language=Haskell]
        take n [] ++ drop n [] = []
        [] ++ [] = []
        [] = [] \end{lstlisting}

    \item Hipótesis:
      \begin{lstlisting}[language=Haskell]
        take n xs ++ drop n xs = xs \end{lstlisting}

    \item Paso inductivo:
      \begin{itemize}
      \item[--] Por demostrar:
	\begin{lstlisting}[language=Haskell]
          take n (x:xs) ++ drop n (x:xs) = (x:xs)
          (x : take n-1 xs) ++ (drop n-1 xs) = (x:xs)
          -- Por definicion de take podemos sacar la x, entonces
          x:(take n-1 xs ++ drop n-1 xs) = (x:xs)
          -- Aplicamos la hipotesis de induccion
          x:xs = x:xs  	\end{lstlisting}
      \end{itemize}
    \end{itemize}

  \item
    \begin{lstlisting}[language=Haskell]
      take m . take n = take (min m n)\end{lstlisting}
    \begin{itemize}
    \item Caso base:
      \begin{lstlisting}[language=Haskell]
        take m . take n [] = take (min m n) []
        take m . [] = take (min m n) []
        [] = take (min m n) []
        -- A partir de la lista vacia construimos el lado derecho.
        take (min n m) [] = take (min n m) []
      \end{lstlisting}
    \item Hipótesis:
      \begin{lstlisting}[language=Haskell]
        take m . take n = take (min m n)
      \end{lstlisting}
    \item Paso Inductivo:\\

      \begin{itemize}
      \item[--] Por demostrar
  	\begin{lstlisting}[language=Haskell]
  	  take m . take n = take (min m n)\end{lstlisting}
      \item[--] $n < m$:
  	\begin{lstlisting}[language=Haskell]
  	  take m . take n (x:xs) = take (min m n) (x:xs)
  	  -- Por la definicion de take
  	  take m . x : take n-1 xs = take (min m n) (x:xs)
  	  -- Volvemos a aplicar la definicion de take
  	  x: take m-1 . take n-1 xs = take (min m n) (x:xs)
  	  -- Aplicamos la hipotesis
  	  x: take (min n-1 m-1) xs = take (min m n) (x:xs)
  	  -- Aplicamos en sentido opuesto la definicion de take
  	  take (min n-1 m-1)+1 (x:xs) = take (min m n) (x:xs)
  	  --Como suponemos que n < m
  	  take n-1+1 (x:xs) = take n (x:xs)
  	  take n (x:xs) = take n (x:xs)\end{lstlisting}
      \item[--] $n > m$:
  	\begin{lstlisting}[language=Haskell]
  	  take m . take n (x:xs) = take (min m n) (x:xs)
  	  -- Por la definicion de take
  	  take m . x : take n-1 xs = take (min m n) (x:xs)
  	  -- Volvemos a aplicar la definicion de take
  	  x: take m-1 . take n-1 xs = take (min m n) (x:xs)
  	  -- Aplicamos la hipotesis
  	  x: take (min n-1 m-1) xs = take (min m n) (x:xs)
  	  -- Aplicamos en sentido opuesto la definicion de take
  	  take (min n-1 m-1)+1 (x:xs) = take (min m n) (x:xs)
  	  --Como suponemos que n < m
  	  take m-1+1 (x:xs) = take m (x:xs)
  	  take m (x:xs) = take m (x:xs)
  	\end{lstlisting}

      \end{itemize}
    \end{itemize}

  \item
    \begin{lstlisting}[language=Haskell]
      map f . take n = take n . map f\end{lstlisting}
      \begin{itemize}
      \item Caso base:
        \begin{lstlisting}[language=Haskell]
          map f . take n [] = take n . map f []
          map f [] = take []
          [] = []\end{lstlisting}
      \item Hipótesis:
        \begin{lstlisting}[language=Haskell]
          map f . take n xs= take n . map f xs\end{lstlisting}
        \item Paso Inductivo:
  	  \begin{itemize}
  	  \item[--] Por demostrar
            \begin{lstlisting}[language=Haskell]
              map f . take n (x:xs)= take n . map f (x:xs)
              map f . x : take (n-1) xs = take n . map f (x:xs)
              f x : map f . take (n-1) (xs) = take n . map f (x:xs)
              -- Por hipotesis de induccion
              f x : take n-1 . map f xs = take n . map f (x:xs)
              -- Aplicamos take de derecha a izquierda
              take n (f x : map f xs) = take n . map f (x:xs)
              -- Aplicamos map de derecha a izquierda
              take n . map f (x:xs) = take n . map f (x:xs)\end{lstlisting}
          \end{itemize}
      \end{itemize}
    \item
      \begin{lstlisting}[language=Haskell]
        filter p . concat = concat . map (filter p)\end{lstlisting}
      Esto es falso verdadero ya que en GCHi indica un error por el
      operador (.); Sin embargo con $\$$ se cumple, por lo que vamos
      a suponer que se usa $\$$ en lugar de (.) para solucionar el
      ejercicio.
      \begin{itemize}
      \item Caso base:
	\begin{lstlisting}[language=Haskell]
          filter p $ concat [] = concat $ map (filter p) []
          filter p $ [] = concat $ []
          [] = []\end{lstlisting}
      \item Hipótesis:
        \begin{lstlisting}[language=Haskell]
          filter p $ concat xss= concat $ map (filter p) xss\end{lstlisting}
        \item Paso Inductivo:
          \begin{lstlisting}[language=Haskell]
            filter p $ concat (xs:xss) = concat $ map (filter p) (xs:xss)
            filter p $ xs ++ concat xss = concat $ map (filter p) (xs:xss)
            -- La demostracion de que el filter distribuye se deja
            -- como ejercicio al lector xD. Suponemos que distribuye
            filter p xs ++ filter p $ concat xss = concat $ map (filter p) (xs:xss)
            -- Por hipotesis de induccion
            filter p xs ++ concat $ map (filter p) xss = concat $ map (filter p) (xs:xss)
            -- Concat al reves
            concat $ filter p xs : map (filter p) xss = concat $ map (filter p) (xs:xss)
            -- Map al reves
            concat $ map (filter p) (xs:xss) = concat $ map (filter p) (xs:xss)
          \end{lstlisting}
  	  \begin{itemize}
  	  \item[--] Por demostrar
  	  \end{itemize}
      \end{itemize}
  \end{itemize}


  \section{Función map}
  Consideremos la siguiente afirmación
  \begin{lstlisting}[language=Haskell]
    map (f . g) xs = map f $ map g xs
  \end{lstlisting}
  \begin{enumerate}[(a)]
  \item ¿Se cumple para cualquier xs? Si es cierta bosqueja la demostración, en
    caso contrario, ¿Qué condiciones se deben pedir sobre xs para que sea cierta?

    Sospechamos fuertemente que si se cumple para cualquier xs, pues hicimos
    varias pruebas, pero no demostramos para todos los tipos.

    \begin{lstlisting}[language=Haskell]
      map (f . g) xs = map f $ map g xs
    \end{lstlisting}
    \begin{itemize}
    \item Caso base:
      \begin{lstlisting}[language=Haskell]
        map (f . g) [] = map f $ map g []
        map f (map g [])
        map f $ map g []
      \end{lstlisting}
    \item Hipótesis
      \begin{lstlisting}[language=Haskell]
        map (f . g) xs = map f $ map g xs
      \end{lstlisting}
    \item Paso inductivo
      \begin{itemize}
      \item [--] Por demostrar
        \begin{lstlisting}[language=Haskell]
          map (f . g) (x:xs) = map f $ map g (x:xs)
        \end{lstlisting}
      \end{itemize}
      \begin{lstlisting}[language=Haskell]
        map (f . g) (x:xs) = map f $ map g (x:xs)
        map f (map g  (x:xs)) = map f $ map g (x:xs)
        map f (g x : map xs) = map f $ map g (x:xs)
        f (g x) : map f (map g xs) = map f $ map g (x:xs)
        -- Hipotesis
        f (g x) : map f $ map g xs = map f $ map g (x:xs)
        map f (g x) : $ map g xs = map f $ map g (x:xs)
        map f $ map g (x:xs) = map f $ map g (x:xs)
      \end{lstlisting}
    \end{itemize}    
    
  \item Intuitivamente, ¿Qué lado de la igualdad resulta más eficiente? ¿Esto es
    cierto incluso en lenguajes con evaluación perezosa? Justifica tu respuesta.

    Intuitivamente, el lado derecho resulta más eficiente, pues solo tendríamos
    que resolver:
    \begin{lstlisting}[language=Haskell]
      map g xs
    \end{lstlisting}
    Y poder pasar el resultado, ya que hacer la composición del otro lado,
    intuitivamente no se ve tan eficiente.

    Para el caso de lenguajes con evaluación perezosa no creemos que sea cierto,
    pues le combiene al lenguaje hacer la composición para ver que pueda proceder
    a aplicarla y posteriormente aplicarla, ya que un lenguaje con evaluación
    perezosa no hace nada hasta que tenga que hacerlo.
  \end{enumerate}

\end{document}
