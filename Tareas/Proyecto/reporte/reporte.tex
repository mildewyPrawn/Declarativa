%%% TeX-command-extra-options: "-shell-escape"

\documentclass[10pt,a4paper]{article}
\usepackage[T1]{fontenc}
\usepackage[sfdefault]{overlock}
\usepackage[spanish]{babel}
\usepackage[utf8]{inputenc}
\usepackage{amsfonts}
\usepackage{amsmath}
\usepackage{graphicx}
\usepackage{hyperref}
\usepackage{lipsum}
\usepackage{listings}
\usepackage{listings}
\usepackage{minted}
\usepackage{pagecolor}
\usepackage{tikz}
\usepackage{xcolor}

\hypersetup{
  colorlinks=true,
  linkcolor=blue,
  filecolor=magenta,
  urlcolor=gunaPink,
}

\definecolor{gunaPink}{RGB}{184,44,93}
\definecolor{gunaBlue}{RGB}{22,28,35}
\definecolor{gunaBlue2}{RGB}{31,40,50}
\definecolor{gunaBlueLite}{RGB}{2,146,176}
\definecolor{gunaOrange}{RGB}{232,134,39}
\definecolor{gunaGray}{RGB}{61,73,81}
\definecolor{gunaString}{RGB}{230,184,184}

\lstset{
  frame=none,
  xleftmargin=2pt,
  stepnumber=1,
  numbers=left,
  numbersep=5pt,
  numberstyle=\ttfamily\tiny\color[gray]{0.3},
  belowcaptionskip=\bigskipamount,
  captionpos=b,
  escapeinside={*'}{'*},
  language=haskell,
  tabsize=2,
  emphstyle={\bf},
  commentstyle=\it,
  stringstyle=\mdseries\rmfamily,
  showspaces=false,
  keywordstyle=\bfseries\rmfamily,
  columns=flexible,
  basicstyle=\small\sffamily,
  showstringspaces=false,
  morecomment=[l]\%,
}


\addtolength{\voffset}{-1cm}
\addtolength{\hoffset}{-0.5cm}
\addtolength{\textwidth}{1cm}
\addtolength{\textheight}{2cm}

\title{Facultad de Ciencias, UNAM.\\Programación declarativa.\\ Proyecto.}
\author{Emiliano Galeana Araujo\\ Juan Alfonso Garduño Solís}
\date{\today}

\begin{document}
\maketitle
\section{Introducción}
\noindent A grandes rasgos nuestro proyecto es un programa que a partir de un
archivo fuente con notas músicales genera ondas de sonido correspondientes a
dichas notas  para después reproducirlas con un programa externo. Propusimos este
tema de proyecto por curiosidad y gusto; curiosidad porque no teníamos mucha idea
de cómo generar sonidos con algún lenguaje de programación y mucho menos con un
lenguaje de programación declarativo, pero sabiamos que seguramente era posible y
queríamos intentarlo, y gusto porque a quién no le gusta la música. Naturalmente
elegimos Haskell porque si nos ibamos a meter en un terreno desconocido, al menos
queríamos conocer un poco el lenguaje, y honestamente fué una buena decisión. A
continuación explicamos las ideas fundamentales para desarrollar el proyecto,
como ejecutar el programa resultante y las conclusiónes que pudimos obtener a
partir de esta experiencia.

\section{Justificación del diseño}
\noindent Para explicar el desarrollo se deben tener en mente ciertos conceptos
de sonido y música. El sonido es una vibración que se propaga, naturalmente en
forma de onda y según la forma que tenga la onda es el sonido que podemos
percibir, por ejemplo la intencidad del sonido está en función de la amplitud
de la onda y el tono del sonido está en función de la frecuencia. Si tomamos una
onda, por ejemplo sinusoidal, y modificamos su frecuencia a 440Hz decimos que la
nota que genera es un \textit{La}, y como no estamos generando sus armónicos
decimos que es una nota pura.

Con esto explicado vamos a seguir el proceso por el que pasa una nota del archivo
de entrada hasta su reproducción. Entonces, cuando invocamos al función principal
\texttt{interactive} lo primero que vamos a hacer es separar el archivo de
entrada en una tupla (t,c), dónde \texttt{t} va a ser la duración de las notas y
\texttt{c} va a ser una cadena que contiene todas las notas del archivo, la
función encargada de esto es \texttt{parseFile} y se auxilia de
\texttt{concater} y \texttt{cleanComment}, a continuación llamamos a la función
principal \texttt{play} con t y el resultado de aplciar la función
\texttt{translate} a \texttt{c}, \texttt{translate} lo que va a hacer es arrojar
una lista de números flotantes sólo para las notas contempladas, por ejemplo, en
nuestro caso tenemos en la lista \texttt{notes} 12 posibles notas con sus
respectivos ordenes, por lo que los 12 primeros elementos de la lista
\texttt{['A'..]} tienen una nota asociada, las primeras 7 para las notas
naturales y las 5 restantes para las notas sostenidas. Con esto terminamos el
proceso de análisis sitnactico del archivo, cabe resaltar que gracias a este
diseño agregar notas se vuelve trivial, solo se deben agregar las notas deseadas
a la lista \texttt{notes} del archivo \textit{Proyecto.hs}.

Los números que tiene un elemento de tipo \texttt{Note}, son el nombre y un
flotante, este le da un orden en la escala musical, por ejemplo \texttt{Do} es la
primer nota, por lo que tiene asociado el número \texttt{0.0}. La razón de usar
flotantes y no enteros es que eso nos evita parsearlo en la parte de
\texttt{Proyecto.wave}.

Una vez que tenemos la lista de notas, podemos producir sonido, para esto, nos
auxiliamos de una fórmula\cite{formula}, la cual nos ayuda a calcular la
frecuencia que tiene que tener una nota musical. Personalmente, nos agrada el
sonido de un piano, así que haciendo un poco de investigación, nos encontramos
con que, el piano al tener 88 teclas, la tecla número 49 es la nota
\texttt{La$_{4}$}, y con esa nueva fórmula, podemos calcular las 88 teclas del
piano\footnote{Sin embargo, esto no es tan sencillo. Ver
  \texttt{\hyperref[sec:TaF]{Trabajo a Futuro}}}






\textbf{Aquí viene la parte difícil, entiendo el código y todo el desmadre pero ya no supe como redactarlo, pero sale hoy}

\section{Ejecución}
\noindent Para ejecutar correctamente el programa se debe de tener el software
\href{https://ffmpeg.org/}{FFmpeg} instalado en la computadora ya que este es el
encargado de reproducir las notas generadas por nuestro programa a través de la
herramienta \texttt{ffplay}, la instalación no debería ser complicada. Para el
desarrollo del proyecto se utilizó \href{https://www.haskell.org/ghc/}{GHC}, así
que se recomienda que el modo interactivo que se utilice para cargar el proyecto
sea \texttt{GHCi}, no garantizamos el correcto funcionamiento del proyecto si se
utiliza un compilador distinto.

Además de tener el archivo de entrada (la canción), el cuál debe estar construido
sigueindo las siguientes reglas:
\begin{itemize}
\item[\textbullet] El archivo puede tener comentarios, las lineas del archivo que
  sean un comentario deben de comenzar con el caractér \textbf{\#}.
  \begin{itemize}
  \item [Ejemplo:] \texttt{\# Este es un comentario}
  \end{itemize}
\item[\textbullet] La primer linea del archivo que no sea un comentario debe de
  ser un número (no necesariamente entero), este número  representa el tiempo en
  segundos que todas las notas van a durar.
  \begin{itemize}
  \item [Ejemplo:] \texttt{0.25}
  \item [Ejemplo:] \texttt{1}
  \end{itemize}
\item[\textbullet] Las lineas siguientes al número del tiempo deben ser
  comentarios o notas en la notación anglosajona musical separadas por un espacio.
  \begin{itemize}
  \item [Ejemplo:] \texttt{C C G G A A G}

    \textit{Este ejemplo representaría las notas
      \texttt{Do Do Sol Sol La La Sol}}
  \end{itemize}
\end{itemize}

Los archivos \texttt{estrellita.je, titanic.je} son ejemplos de archivos con una
entrada bien construida y cofifican la canción que su nombre lo indica. Suenan
relativamente bien, pero no tanto.

Existen dos maneras de ejecutar el programa, con el intérprete y compilando.

\subsection{Intérprete}
Una vez cumplidos los requisitos anteriores (el archivo bien hecho) se debe abrir
una terminal en un directorio que contenga los tres módulos del proyecto, los
cuales llevan por nombre \texttt{Main.hs, Music.hs, Proyecto.hs} y el archivo que
contiene la canción a ejecutar, vamos a decir que este archivo es
\texttt{estrellita.je}.

En la terminal vamos a ingresar al modo interactivo de GHC con el comando
\texttt{gchi Main.hs}, ya que estamos dentro con el comando
\texttt{interactive ''estrellita.je''} se ejecuta el programa y se reproduce la
cancion al cabo de unos segundos.

\subsection{Compilar (make)}
Para esta opción, se requiere el paquete
\href{https://www.archlinux.org/packages/core/x86\_64/make/}{make} y las opciones
para nuestro programa son:

\begin{itemize}
\item \texttt{make all}, el cuál nos muestra todas las opciones (Las que se
  muestran a continuación).
\item \texttt{make compile}, el cual compila el proyecto.
\item \texttt{make run FILE=''estrellita.je''}, el cual ejecuta el programa y pasa
  como argumento el archivo \texttt{estrellita.je}.
\item \texttt{make clean}, el cual limpia los archivos que se generan cuando se
  compila, así como el ejecutable y el archivo que interpreta \texttt{ffplay}.
\end{itemize}

Adionalmente, se puede compilar el proyecto con las instrucciones que se
encuentran en el \texttt{Makefile} (En caso de que no se tenga el paquete
\texttt{make} o se tenga desconfianza de lo que este pueda hacer).

\section{Trabajo a Futuro}
\label{sec:TaF}
Existen varias bibliotecas\cite{paul} que se pueden utilizar para generar un
sonido más puro, incluso usando varios instrumentos; Sin embargo nos fue
complicado usarlas, por cuestiones de \texttt{Cabal}, o del reproductor de música
(Se necesita un reproductor que pueda interpretar archivos \texttt{MIDI}). Así
que portar el proyecto para que pueda producir sonidos más puros sería un buen
avance.

Otra cosa es que cualquier música que queramos reproducir, las notas tienen un
tiempo defindo y es el mismo (La primer línea de los archivos \texttt{*.je}),
poder cambiar el sonido en distintas notas sería un buen avance para producir
música más del estilo que conocemos.

Lo último es que, debido a cómo generamos el sonido, algunos archivos pueden
llegar a pesar mucho (Como el \texttt{cancion.bin} de \texttt{titanic.je}), esto
por el número de ondas que generamos, como se dijo anteriormente, ese número fue
el más aproximado a un sonido grave, reducir el tamaño de los archivos de salida
sería algo ideal para una segunda versión del proyecto.

\section{Conclusiones}
\noindent Normalmente como programadores no salimos de ciertas zonas de confort
(paradigmas) porque son con las que nos instruyeron o son las maneras más
naturales de pensar una solución a los problemas que se nos pueden presentar, sin
embargo hay formas de solucionar los mismos problemas con distintos enfoques.
Para realizar un ejercicio como el presentado en este proyecto lo último que
alguien normalmente pensaría sería programarlo en Haskell o cualquier otro
lenguaje declarativo porque es difícil ver el camino a seguir, pero es posible y
tal vez hasta es más fácil y rápido que el camino habitual. Si bien el proyecto
honestamente no tenga mucha utilidad, es el fruto del uso de practicamente todo
lo revisado en el curso, desde utilizar elementos que nos brinda el lenguaje y el
paradigma como listas por comprención, caza de patrones, mónadas u operadores
fold, hasta estrategias (mañas) para solucionar problemas con cierto patron.
Claro que además de la capacidad de modelar el problema en la computadora, en
este caso particular modelar las ondas de sonido, dicha capacidad adquirida a lo
largo de la carrera.

Entonces, en conclusión podemos decir que el proyecto fue divertido, aprendimos
varias cosas (No solamente de computación), incluso estructuras de datos que
nunca habíamos usado en un entorno declarativo:

\begin{minted}{haskell}
  translate (x:xs) =
  if isAlpha x
  then
  let
  posible_note = Map.lookup x mapi
  in
  case posible_note of
  Just (note, ord) -> [(ord + 40)] ++ translate xs
  Nothing -> [] -- omite las letras que no estn.
  else translate xs
\end{minted}

Podría ser algo que desarrollemos mejor en un futuro, no es de mucha utilidad,
ya que existen mejores maneras de crear música, incluso con código, más aún,
herramientas que utilizan haskell\cite{tidal}, pero sin duda es algo distinto a
todo lo que habíamos hecho antes en la carrera.

\bibliography{bib} {
  \nocite{paul}
  \nocite{tidal}
  \nocite{u2}
  \nocite{otro}
}
\bibliographystyle{ieeetr}

\end{document}
