\documentclass[spanish,12pt,letterpaper]{article}

\usepackage[english]{babel}
\usepackage[table]{xcolor}
\usepackage[utf8]{inputenc}
\usepackage{amsmath}
\usepackage{amssymb}
\usepackage{authblk}
\usepackage{csquotes}
\usepackage{enumerate}
\usepackage{float}
\usepackage{forest}
\usepackage{geometry}
\usepackage{graphicx}
\usepackage{listings}
\usepackage{xcolor}

\geometry{
  a4paper,
  total={170mm,257mm},
  left=20mm,
  top=20mm,
}

\definecolor{codegreen}{rgb}{0,0.6,0}
\definecolor{codegray}{rgb}{0.5,0.5,0.5}
\definecolor{codepurple}{rgb}{0.58,0,0.82}
\definecolor{backcolour}{rgb}{0.95,0.95,0.92}

\lstdefinestyle{mystyle}{
  backgroundcolor=\color{backcolour},
  commentstyle=\color{codegreen},
  keywordstyle=\color{orange},
  numberstyle=\tiny\color{codegray},
  stringstyle=\color{codepurple},
  basicstyle=\ttfamily\footnotesize,
  breakatwhitespace=false,
  breaklines=true,
  captionpos=b,
  keepspaces=true,
  numbers=right,
  numbersep=5pt,
  showspaces=false,
  showstringspaces=false,
  showtabs=false,
  tabsize=2
}

\lstset{style=mystyle}

\title{Programación declarativa. Tarea 3\\
  \Huge{Bringing you into the fold}}
\author{Juan Alfonso Garduño Solís\\
  Emiliano Galeana Araujo}
\affil{Facultad de ciencias, UNAM}
\date{Fecha de entrega: Jueves 12 de marzo de 2020}

\begin{document}

\maketitle

\section{Propiedades}
\begin{enumerate}[(a)]
\item
  \begin{lstlisting}[language=Haskell]
    foldr f e . map g = foldr (f . g) e
  \end{lstlisting}
  Lo probaremos usando la ley de fusión.

  Recordemos que
  \begin{lstlisting}[language=Haskell]
    map g = foldr ((:) . g) []
  \end{lstlisting}
  Usando inducción, vemos que cuando tenemos la lista vacía la función
  \texttt{foldr f e []} nos regresa \texttt{e} por definición. Por lo que para la
  lista no vacía tenemos lo siguiente.
  \begin{lstlisting}[language=Haskell]
    foldr f e (g (x:xs)) = h x (foldr f e xs)
    -- La parte izquierda se reduce a...
    f (g x) (foldr f a xs)
  \end{lstlisting}
  Por lo que podemos definir \texttt{h x y = f (g x) y} con la ley de fusión.

  y \texttt{h} se puede ver como una composición (\texttt{h = f . g}), por lo que
  tenemos
  \begin{lstlisting}[language=Haskell]
    foldr f e . map g = foldr (f . g) e
  \end{lstlisting}
  Para cualquier lista finita.

\item
  \begin{lstlisting}[language=Haskell]
    foldl f e xs = foldr (flip f) e (reverse xs)
  \end{lstlisting}
  Rescribiremos
  \begin{lstlisting}[language=Haskell]
    g = flip f
  \end{lstlisting}
  Por lo que provaremos:
  \begin{lstlisting}[language=Haskell]
    foldl f e xs = foldr g e (reverse xs)
  \end{lstlisting}
  Por inducción para listas finitas.
  \begin{itemize}
  \item Caso base ([ ]):
    \begin{lstlisting}[language=Haskell]
      foldl f e [] = e
      -- Por definicion de caso base de foldl
    \end{lstlisting}
    Por otro lado tenemos:
    \begin{lstlisting}[language=Haskell]
      foldr g e (reverse []) = foldr g e []
      -- Por definicion de reverse []
      foldr g e [] = e
      -- Por definicion de caso base de foldr
    \end{lstlisting}
  \item Hipótesis
    \begin{lstlisting}[language=Haskell]
      foldl f e xs = foldr g e (reverse xs)
    \end{lstlisting}
  \item Paso inductivo
    \begin{itemize}
    \item [--] Por demostrar
      \begin{lstlisting}[language=Haskell]
        foldl f e (x:xs) = foldr g e (reverse (x:xs))
      \end{lstlisting}
    \end{itemize}
    \begin{lstlisting}[language=Haskell]
      foldl f e (x:xs) = foldl f (f e x) xs
      -- Por definicion de foldl
      foldl f (f e x) xs = foldr g (f e x) (reverse xs)
      -- Por induccion
    \end{lstlisting}
    Por otro lado tenemos:
    \begin{lstlisting}[language=Haskell]
      foldr g e (reverse (x:xs)) = foldr g e (reverse xs ++ [x])
      -- Por definicion de reverse
      foldr g e (reverse xs ++ [x]) = foldr g (foldr g e [x]) (reverse xs)
      -- Por c)
      foldr g (foldr g e [x]) (reverse xs) = foldr g (f e x) (reverse xs)
    \end{lstlisting}
    Como llegamos a lo mismo, afirmamos que se cumple para cualquier lista. Solo
    recordemos cambiar cambiar a \texttt{g}.
  \end{itemize}

\item
  \begin{lstlisting}[language=Haskell]
    foldr f e (xs ++ ys) = foldr f (foldr f e ys) xs
  \end{lstlisting}
  Tomemos en cuenta la siguiente igualdad.
  \begin{lstlisting}[language=Haskell]
    foldr f e (xs ++ ys) = foldr f e (foldr (:) xs ys)
  \end{lstlisting}
  Recordemos la ley de fusión que nos dice que:
  \begin{lstlisting}[language=Haskell]
    f . foldr g a = foldr h b
    -- Donde
    -- f a = b.
    -- f (g x y) = h x (f y)
    -- f es estricta.
  \end{lstlisting}
  Aplicamos la ley de la siguiente manera:
  \begin{lstlisting}[language=Haskell]
    f = foldr f e
    g = (:)
    h = f -- f es la f del problema.
    a = ys
    b = foldr f e ys
  \end{lstlisting}
  Tomamos las siguientes cadenas de igualdades, basándonos en las definiciones
  anteiores:
  \begin{lstlisting}[language=Haskell]
    f a = foldr f e a = foldr f e ys = b

    h x (f y) = f x (foldr f e y) = foldr f e (x:y) = foldr f e ((:) x y) = foldr f e (g x y) = f (g x y).
  \end{lstlisting}
  Con lo anterior, sabemos que la ley de fusión es aplicable. Por lo que al
  aplicarla, tenemos que
  \begin{lstlisting}[language=Haskell]
    foldr f e (xs ++ ys) = foldr f e (foldr (:) ys xs) = foldr f (foldr f e ys)xs
  \end{lstlisting}
\end{enumerate}

\section{Árboles Binarios}

Consideramos el siguiente tipo de dato algebráico en Haskell para definir árboles
binarios.
\begin{lstlisting}[language=Haskell]
  data Tree a = Void | Node (Tree a) a (Tree a)
\end{lstlisting}

Y la función \texttt{foldT} que define el operador de plegado para la estructura
\texttt{Tree}, definido como sigue:

\begin{lstlisting}[language=Haskell]
  foldT :: (b -> a -> b -> b) -> b -> Tree a -> b
  foldT _ v Void = v
  foldT f v (Node t1 r t2) = f t1' r t2'
  where t1' = foldT f v t1
  t2' = foldT f v t2
\end{lstlisting}

\begin{enumerate}
\item Da en términos de una función \texttt{h} el patrón de encapsulado por el
  operador \texttt{foldT}.
\begin{lstlisting}[language=Haskell]
h Void = v
h (Node t1 r t2) = f (h t1) r (h t2)\end{lstlisting}

\item Enuncia y demuestra la propiedad Universal del operador \texttt{foldT},
  basándose en la Propiedad Universal vista en clase sobre el operador
  \textcolor{orange}{foldr} de listas.
\begin{itemize}
\item Propiedad Universal:\\
\begin{align*}
h\;Void\; = \;v \qquad \qquad \qquad \qquad \;\;\;\;\;\;\\
h\;Node\;t_1\;r\;t_2\;=\;f\;(h\;t_1)\;r\;(h\;t_2)\;&\;\Leftrightarrow\;h\;t\;=\;foldT\;f\;\digamma
\end{align*}
\item Demostración: \\

\begin{itemize}
\item[$\Rightarrow$)]
\begin{align*}
h\;Void\; = \;v \qquad \qquad \qquad \qquad \;\;\;\;\;\;\\
h\;Node\;t_1\;r\;t_2\;=\;f\;(h\;t_1)\;r\;(h\;t_2)\;&\;\Rightarrow\;h\;t\;=\;foldT\;f\;\digamma
\end{align*}

\begin{itemize}
\item Caso base:\\
	\begin{align*}
	h\;Void\;&= v \\
	h\;Void\;&=\;foldT\;f\;v\;Void
	\end{align*}
\item Hipótesis de inducción:
	\begin{align*}
	h\;t=foldT\;f\;v\;t
	\end{align*}
\item Paso inductivo:
\begin{align*}
h\;Node\;t_1\;r\;t_2 &=foldT\;f\;v\;Node\;t_1\;r\;t_2\\
f\;(h\;t_1)\;r\;(h\;t_2) &=foldT\;f\;v\;Node\;t_1\;r\;t_2
\qquad \qquad\qquad\qquad
\end{align*}
Aplicamos la hipotesis de inducción
\begin{align*}
f\;(foldT\;f\;v\;t_1)\;r\;(foldT\;f\;v\;t_2)&=foldT\;f\;v\;Node\;t_1\;r\;t_2\qquad\qquad\qquad\qquad
\end{align*}
Por definición de $foldT$
\begin{align*}
foldT\;f\;v\;Node\;t_1\;r\;t_2&=foldT\;f\;v\;Node\;t_1\;r\;t_2\qquad\qquad\qquad
\end{align*}
\end{itemize}
\end{itemize}
\end{itemize}
\end{enumerate}

\section{Función \texttt{scanr}}
Calcula una definición eficiente para \textcolor{orange}{scanr} partiendo de la
siguiente:

\begin{lstlisting}[language=Haskell]
  scanr r f e = map (foldr f e) . tails
\end{lstlisting}

\section{Función \texttt{cp}}
Considera la siguiente definición de la función \texttt{cp} que calcula el
producto cartesiano.

\begin{lstlisting}[language=Haskell]
  cp :: [[a]] -> [[a]]
  cp = foldr f e
\end{lstlisting}

\begin{enumerate}
\item En la definición anterior, ¿Quiénes son \texttt{f} y \texttt{e}?
\begin{itemize}
\item \texttt{f}:
		\begin{lstlisting}[language=Haskell]
f :: [a] -> [[a]] -> [[a]]
f [] _ = []
f (x:xs) l = (map (x:) l) ++ (f xs l)\end{lstlisting}
\item \texttt{e}: \texttt{[[]]}
\end{itemize}

\item Dada la siguiente ecuación
  \begin{lstlisting}[language=Haskell]
    length . cp = product . map length
  \end{lstlisting}
  en donde \textcolor{orange}{length} calcula la longitud de una lista y
  \textcolor{orange}{product} regresa el resultado de la multiplicación de todos
  los elementos de una lista. Demuestra que la ecuación es cierta, para esto es
  necesario reescribir ambos lados de la ecuación como instancias de \texttt
  {orange}{foldr} y ver que son idénticas.
\end{enumerate}

\section{Parte extra}
En una granja con mucho folklore se discute acerca del siguiente
razonamiento.  El día que nace un becerro, cualquiera lo puede cargar con
facilidad.  Y los becerros no crecen demasiado en un día,  entonces si puedes
cargar a un becerro un día,  lo puedes cargar también al día siguiente,
siguiendo con este razonamiento entonces también debería serte posible cargar al
becerro el día siguiente y el siguiente y así sucesivamente. Pero después de un
año, el becerro se va a convertir en una vaca adulta de 1000kg algo que
claramente ya no puedes cargar.

Este es un razonamiento inductivo, la base es el día que el becerro nace,
suponemos cierto que se puede cargar en el día \textit{n} de vida del becerro y
si se puede cargar ese día, como no crece mucho en un día entonces también se
puede cargar en el día \textit{n+ 1}.  Pero claramente la conclusión es falsa.

Para este ejercicio hay dos posibles soluciones,  la primera es indicar en donde
está el error en el razonamiento inductivo o la segunda es cargar una vaca adulta
así demostrando que el argumento es correcto.\\

\noindent En este caso queremos hacer inducción sobre el número de días transcurridos tras el día de nacimiento del becerro con el objetivo de
demostrar una propiedad sobre el peso de los becerros: Su
peso no es el suficiente para evitar cargarlo. Vamos a ver los números como
conjuntos de becerros, de manera que el primer conjunto sería el conjunto que
tiene al becerro con un día de nacido, el segundo conjunto sería el conjunto
que tiene al becerro con un día de nacido y al becerro con dos días de nacido y
así sucesivamente.\\
De una manera más formal la demostración enunciada sería de la
siguiente manera:\\

\begin{itemize}
\item \textbf{Caso base:}\\
Es posible cargar a todos los becerros que están en el conjunto que solo tiene
al becerro con un día de nacido. Se cumple.

\item \textbf{Hipótesis de inducción:}\\
Es posible cargar a todos los becerros que están en el conjunto $n$-esímo, el
cual tiene al becerro con un día de nacido, al que tiene dos, al que tiene tres
y así sucesivamente hasta n.

\item \textbf{Paso inductivo:}\\
Sabemos por hipótesis que es posible cargar a todos los becerros del conjunto
con $n$ becerros. Para demostrar que se cumple para el conjunto con $n+1$
becerros hacemos lo siguiente:\\
Formamos a todos los becerros de menor a mayor respecto al número de días que
tienen de nacidos, la fila tiene $n+1$ becerros, si quitamos al que tiene
$n+1$ días de nacido nos queda una fila de $n$ becerros y por lo tanto el conjunto que tiene a los becerro de la fila cumple
la hipótesis, pero ¿qué hay del becerro que quitamos?, lo podemos devolver a la
fila y ahora quitamos por ejemplo el primero, entonces de nuevo
tenemos una fila con $n$ a la que podemos aplicar la hipótesis de inducción y
concluir que si podemos cargar a todos los becerros del conjunto con $n$
becerros, entonces también podemos cargar a todos los becerros del conjunto con
$n+1$ becerros.
\end{itemize}

\noindent La demostración parece correcta, sin embargo hay un agujero en el paso
inductivo; de un conjunto con $n+1$ becerros hacemos dos conjuntos, uno sin el
último: $\{1,2,3,4,\dots,n\}$ y otro sin el primero: $\{2,3,4,\dots,n,n+1 \}$,
el argumento funciona porque aplicamos la hipótesis a los primeros $n$ becerros
y a los últimos $n$ becerros, si podemos cargar a los primeros $n$ y también
podemos cargar a los $n$ últimos, entonces podemos cargar a los $n+1$ becerros,
¿no?. Pues no, supongamos $n=1$, entonces $n+1=2$, por tanto el conjunto sobre
el que queremos demostrar es: $\{1,2\}$, el conjunto con los $n$ primeros
becerros es: $\{1\}$ y el de los últimos es $\{2\}$, el argumento ahora sería:
si podemos cargar al primer becerro y podemos cargar al último, podemos cargar
a todos los becerros del conjunto. Ya no suena tan bien, el problema es que ya
no tenemos becerros en la intersección de los primeros n y los últimos n, y
aunque la intuición nos dice que sí deberíamos ser capaces de cargar un becerro
de dos días de nacido, la inducción no funcióna por intuición, y como este caso rompe la cadena de implicaciones de la
inducción, no podemos afirmar nada de los siguientes becerros.\\
El problema entonces es que no se puede demostrar para $n+1=2$.


\end{document}
